\section{Integration of measurable functions}
Through this chapter $(X,\A,\mu)$ will be some measure space. Recall that $\mathcal M^+(\A)$
$[\mathcal M^+_{\bar\R}(\A)]$ are the
$\A$-measurable positive functions and $\mathcal E(\A)$ $[\mathcal E^+_{\bar\R}(\A)]$ are the positive and simple functions.

The fundamental idea of \textit{Integration} is to measure the area between the graph
of the function and the abscissa. For positive simple functions $f\in\mathcal E^+(\A)$ in standard 
representation, this is done easily
\begin{align}
    \text{if  } f=\sum_{i=0}^M y_i \mathds{1}_{A_i}\in \E^+(\A)\hspace{4mm} \text{then  } \sum_{i=0}^{M}y_i\mu(A_i)
\end{align}
would be the $\mu$-area enclosed by the graph and the abscissa. We note that 
the representation of $f$ should not impact the integral of $f$.

\begin{lemma}
    Let $\sum_{i=0}^{M} y_i\one_{A_i} =\sum_{k=0}^{N} z_k\one_{B_k} $ be two standard representations
    of the same function $f\in\E^+(\A)$. Then 
    \begin{align}
        \sum_{i=0}^{M} y_i\mu(A_i) =\sum_{k=0}^{N} z_k\mu(B_k).
    \end{align}
\end{lemma}


\begin{definition}
    Let $f=\sum_{i=0}^{M} y_i\one_{A_i}\in\E^+(\A)$ be a simple function in standard representation.
    Then the number
    \begin{align}
    I_\mu(f) = \sum_{i=0}^{M} y_i\mu(A_i)\in[0,\infty]    
    \end{align}
    (which is independent of the representation of $f$) is called the $\mu$-integral of $f$.
\end{definition}

\begin{proposition}
    Let $f,g\in\E^+(\A)$. Then
    \begin{enumerate}
        \item[(i)] $I_\mu(\one_A)=\mu(A)\hspace{3mm}\forall A\in\A.$
        \item[(ii)] $I_\mu(\lambda f)=\lambda I_\mu(f)\hspace{3mm} \forall \lambda\geq 0.$
        \item[(iii)] $I_\mu(f+g) = I_\mu(f)+I_\mu(g).$
        \item[(iv)]  $f\leq g\Rightarrow I_\mu(f)\leq I_\mu(g).$
    \end{enumerate}
\end{proposition}

In theorem 8.8 we saw that we could for every $u\in\M^+(\A)$ write it as an increasing limit
of simple functions. By corollary 8.10, the suprema of simple functions are again measurable, so that 
\begin{align*}
    u\in\M^+(\A)\Leftrightarrow u=\sup_{n\in\N} f_n,\hspace{3mm} f\in\E^+(\A),\hspace{3mm} f_n\leq f_{n+1}\leq\ldots.
\end{align*}
We will use this to "inscribe" simple functions (which we know how to integrate) below the graph of a 
positive measurable function $u$ and exhaust the $\mu$-area below $u$.
\begin{definition}
    Let $(X,\A,\mu)$ be a measure space. The $(\mu)$-integral of a positive function $u\in\M_{\bar\R}^+(\A)$ is given by 

    \begin{equation}
        \label{eq:int}
        \int ud\mu=\sup\left\{ I_\mu(g):g\leq u,\omm g\in\E^+(\A) \right\}\in[0,+\infty].
    \end{equation}
    If we need to emphasize the \textit{integration variable}, we write $\int u(x)\mu(dx).$
    The key observation is that the integral $\int\ldots d\mu$ extends $I_\mu.$

\end{definition}
\begin{lemma}
    For all $f\in\E^+(\A)$ we have $\int fd\mu = I_\mu(f).$
\end{lemma}

The next theorem is one of many convergence theorems. It shows that we could have defined \ref*{eq:int} using any increasing
sequence $f_n\uparrow u$ of simple functions $f_n\in\E^+(\A).$

\begin{theorem}(\underline{Beppo Levi})
    Let $(X,\A,\mu)$ be a measure space. For an increasing sequence of functions 
    $(u_n)_{n\in\N}\subset \M_{\bar\R}^+(\A)$, $0\leq u_n\leq u_{n+1}\leq\ldots$, we have for the supremum $u=\sup_{n\in\N} u_n\in\M_{\bar\R}^+(\A)$
    and 

    \begin{align}
        \int \sup_{n\in\N} u_nd\mu =\sup_{n\in\N} \int  u_nd\mu.
    \end{align}
Note we can write $\lim_{n\rightarrow \infty}$ instead of $\sup_{n\in\N}$ as the supremum of an increasing sequence
is its limit. Moreover, this theorem holds in $[0,+\infty]$, so the case $+\infty = +\infty$ is possible. 
\end{theorem}

\begin{corollary}
    Let $u\in \M_{\bar\R}^+(\A)$. Then \begin{align*}
        \int ud\mu = \lim_{n\rightarrow \infty} \int f_n d\mu
    \end{align*}
    holds for every sequence $(f_n)_{n\in\N}\subset \E^+(\A)$ with $\lim_{n\rightarrow\infty} f_n=u.$

\end{corollary}


\begin{proposition} (of integral)
    Let $u,v\in\M_{\bar\R}^+(\A)$. Then
    \begin{enumerate}
        \item[(i)] $\int \one_A d\mu = \mu(A)\hspace{3mm}\forall A\in\A.$
        \item[(ii)] $\int \alpha ud\mu = \alpha\int u d\mu \hspace{3mm} \forall \alpha\geq 0.$
        \item[(iii)] $\int u+v d\mu = \int u d\mu +\int v d\mu.$
        \item[(iv)]  $u\leq v\Rightarrow \int u d\mu\leq \int v d\mu.$
    \end{enumerate}
\end{proposition}

\begin{corollary}
    Let $(u_n)_{n\in\N}\subset \M_{\bar\R}^+(\A).$ Then $\sum_{n=1}^\infty u_n$ is measurable and we have
    \begin{align*}
        \int \sum_{n=1}^\infty u_n d\mu = \sum_{n=1}^\infty \int u_nd\mu
    \end{align*}
    (including the possibility $+\infty = +\infty.$)
\end{corollary}

\begin{theorem}(\underline{Fatou})
    Let $(u_n)_{n\in\N}\subset \M_{\bar\R}^+(\A)$ be a sequence of positive measurable functions. Then 
$u=\liminf_{n\rightarrow\infty} u_n$ is measurable and 

\begin{align}
    \int\liminf_{n\rightarrow\infty}u_n d\mu =\liminf_{n\rightarrow\infty}\int u_n d\mu 
\end{align}
\end{theorem} 

\section{Integrals of Measurable Functions}
We have defined our integral for positive measurable functions, i.e. functions in \(\mathcal{M}^{+}(\mathscr{A})\). To extend our integral
to not only functions in \(\mathcal{M}^{+}(\mathscr{A})\) we first notice that
\begin{align}
    u \in \mathcal{M}_{\overline{\mathbb{R}}}(\mathscr{A}) \Leftrightarrow 
    u = u^+ - u^-, \ u^+, u^-\in \mathcal{M}^{+}_{\overline{\mathbb{R}}},
\end{align}
i.e. that every measurable function can be written as a sum of \textbf{positive} measurable functions.

\begin{definition}[\(\mu\)-integrable]
    A function \(u:X \rightarrow \overline{\mathbb{R}}\) on \((X, \mathscr{A}, \mu)\) is \(\mu\)-\emph{integrable}, if it is 
    \(\mathscr{A}/\mathscr{B}(\overline{\mathbb{R}})\)-measurable and if \(\int u^+ d\mu, \int u^-d\mu < \infty\) (recall the definition
    for the integral of positive measurable functions). Then
    \begin{align}
        \int ud\mu \eqdef \int u^+d\mu - \int u^-d\mu \in (-\infty, \infty)
    \end{align}
    is the (\(\mu\)-)\emph{integral} of u. We write \(\mathcal{L}^1(\mu)\) for the set of all real-valued \(\mu\)-integrable functions
    \footnote{In words, we extend our integral to \cancel{positive} measurable functions by noticing that we can write every measurable 
    function as a sum of positive measurable functions, something that we do know how to integrate. We don't want to run into the problem
    of \(\infty - \infty\), thus we require the integral of the positive and negative parts to both (separately) be less than infinity.}.
\end{definition}
\begin{theorem}
    Let \(u\in \mathcal{M}_{\overline{\mathbb{R}}}(\mathscr{A})\), then the following conditions are equivalent:
    \begin{enumerate}[label=(\roman*)]
        \item \(u \in \mathcal{L}^{1}_{\overline{\mathbb{R}}}(\mu)\).
        \item \(u^+, u^- \in \mathcal{L}^{1}_{\overline{\mathbb{R}}}(\mu)\).
        \item \(\vert u\vert \in \mathcal{L}^{1}_{\overline{\mathbb{R}}}(\mu)\).
        \item \(\exists w \in \mathcal{L}^{1}_{\overline{\mathbb{R}}}(\mu)\) with \(w\geq 0\) s.t. \(\vert u \vert \leq w\).
    \end{enumerate}
\end{theorem}
\begin{theorem}[Properties of the \(\mu\)-integral] The \(\mu\)-integral is: \textbf{homogeneous, additive}, and:
    \begin{enumerate}[label=(\roman*)]
        \item \(\min\left\{u,v\right\}, \max\left\{u,v\right\} \in \mathcal{L}^{1}_{\overline{\mathbb{R}}}(\mu)\) \itemtag{\emph{lattice property}}
        \item \(u\leq v \Rightarrow \int ud\mu \leq \int vd\mu\) \itemtag{\emph{monotone}}
        \item \(\Big\vert \int ud\mu \Big\vert \leq \int \vert u\vert d\mu\) \itemtag{\emph{triangle inequality}}
    \end{enumerate}
\end{theorem}
\begin{remark}
    If \(u(x) \pm v(x)\) is defined in \(\overline{\mathbb{R}}\) for all \(x\in X\) then we can exclude \(\infty - \infty\) and the theorem above just says that 
    the integral is linear:
    \begin{align}
        \int (au + bv)d\mu = a\int ud\mu + b\int vd\mu.
    \end{align}
    This is always true for real-valued \(u,v\in\mathcal{L}^{1}(\mu) = \mathcal{L}^{1}_{\mathbb{R}}(\mu)\), making \(\mathcal{L}^{1}(\mu)\) a vector space with 
    addition and scalar multiplication defined by
    \begin{align}
        (u + v)(x) \eqdef u(x) + v(x), \ (a\cdot u)(x) \eqdef a\cdot u(x),
    \end{align}
    and
    \begin{align}
        \int ... d\mu: \mathcal{L}^{1}(\mu) \rightarrow \mathbb{R}, \ u \mapsto \int ud\mu,
    \end{align}
    is a \textbf{positive linear functional}.
\end{remark}

