\section{Null sets and the Almost Everywhere (lecture 08, 05. Feb.)}
\begin{definition}
    A (\(\mu\)-)null set \(N\in\mathcal{N}_{\mu}\) is a measurable set \(N\in\mathscr{A}\) satisfying
    \begin{align}
        N\in\mathcal{N}_{\mu} \Longleftrightarrow N\in\mathscr{A} \text{ and } \mu(N) = 0.
    \end{align}
    This can be used generally about a `statement' or `property', but we will be interested in questions like 
    `when is \(u(x)\) equal to \(v(x)\)', and we answer this by saying
    \begin{align}
        u=v \ a.e. \Leftrightarrow \left\{ x: u(x) \neq v(x) \right\} \text{ is (contained in) a }\mu\text{-null set.}, 
    \end{align}
    i.e.
    \begin{align}
        u=v \ \  \mu\text{-a.e.} \Leftrightarrow \mu\left( \left\{ x: u(x) \neq v(x) \right\} \right) = 0.
    \end{align}
    The last phrasing should of course include that the set \( \left\{ x: u(x) \neq v(x) \right\}\) is in \(\mathscr{A}\).
\end{definition}
\begin{theorem} \label{th:11.2}
    Let \(u\in \mathcal{M}_{\overline{\mathbb{R}}}(\mathscr{A})\), then:
    \begin{enumerate}[label=(\roman*)]
        \item \(\int\vert u\vert d\mu = 0 \Leftrightarrow \vert u\vert = 0 \text{ a.e. } \Leftrightarrow \mu\left\{u\neq0\right\} = 0\),
        \item \(\mathbbm{1}_{N}u\in \mathcal{L}^{1}_{\overline{\mathbb{R}}}(\mu)\) \ \(\forall \ N\in\mathcal{N}_{\mu}\),
        \item \(\int_Nud\mu = 0\).
    \end{enumerate}
    (i) is really useful, later we will define \(\mathcal{L}^p\) and the \(||\cdot||_p\)-(semi)norm. Then (i) means that if we have a
    sequence \(u_n\) converging to \(u\) in the \(||\cdot ||_p\)-norm then \(u_n(x) = u(x)\) a.e.
\end{theorem}
\begin{corollary}
    Let \(u=v \ \mu \text{-a.e.}\) Then
    \begin{enumerate}[label=(\roman*)]
        \item \(u,v \geq 0 \Rightarrow \int ud\mu = \int vd\mu\),
        \item \(u\in\mathcal{L}^{1}_{\overline{\mathbb{R}}}(\mu) \Rightarrow v\in\mathcal{L}^{1}_{\overline{\mathbb{R}}}(\mu)\) and \(\int ud\mu = \int vd\mu\).
    \end{enumerate}
\end{corollary}
\begin{corollary}
    If \(u\in\mathcal{M}_{\overline{\mathbb{R}}}(\mathscr{A})\), \(v\in\mathcal{L}^{1}_{\overline{\mathbb{R}}}(\mu)\) and \(v\geq0\) then
    \begin{align}
        \vert u\vert \leq v \text{ a.e. } \Rightarrow u\in\mathcal{L}^{1}_{\overline{\mathbb{R}}}(\mu).
    \end{align}
\end{corollary}
\begin{proposition}[Markow inequality]
    For all \(u\in\mathcal{L}^{1}_{\overline{\mathbb{R}}}(\mu), \ A\in\mathscr{A}\) and \(c>0\)
    \begin{align}
        u\left(\left\{ \vert u\vert \geq c \right\} \cap A\right) \leq \frac{1}{c}\int_{A}\vert u\vert d\mu,
    \end{align}
    if \(A=X\), then (obviosly)
    \begin{align}
        u\left\{ \vert u\vert \geq c \right\} \leq \frac{1}{c}\int \vert u\vert d\mu.
    \end{align}
\end{proposition}    
\ifdetailed
\begin{corollary}
    If \(u\in\mathcal{L}^{1}_{\overline{R}}(\mu)\), then \(\mu\) is a.e. \(\mathbb{R}\)-vaued. In particular, we can find a version
    \(\tilde{u}\in \mathcal{L}^{1}(\mu)\) s.t. \(\tilde{u}=u\) a.e. and \(\int\tilde{u}d\mu = \int ud\mu\)
\end{corollary}
\fi

\subsection*{Completions of measure spaces}
\begin{definition}
    A measure space \(\left(X,\mathscr{B}, \mu\right)\) is called \textbf{complete} if whenever \(A\in\mathscr{B} \text{ and } \mu(A) =0\), we have \(B\in\mathscr{B} \ \forall B\subset A\).
\end{definition}
\begin{remark}
    Any measure space can be completed as follows: \\ 
    Let \(\bar{\mathscr{B}}\) be the \(\sigma\)-algebra generated by \(\mathscr{B}\) and all sets \(B\subset X\) s.t. there exists \(A\in\mathscr{B}\)
    with \(B\subset A\) and \(\mu(A)=0\).
\end{remark}
\begin{proposition}
    The \(\sigma\)-algebra \(\bar{\mathscr{B}}\) can also be described as follows:
    \begin{align}
        \bar{\mathscr{B}} \eqdef \left\{ B\subset X: A_1\subset B\subset A_2\text{ for some }A_1,A_2\in\mathscr{B}\text{ with }\mu(A_2\backslash A_1)=0 \right\},
    \end{align}
    with \(B,A_1,A_2 \) as above, we define 
    \begin{align}
        \bar{\mu} \eqdef \mu(A_1) = \mu(A_2)
    \end{align}
    Then \(\left( X,\bar{\mathscr{B}}, \bar{\mu} \right)\) is a complete measure space.
\end{proposition}

%%% Proof %%%

\begin{definition}
    If \(\mu\) is a Borel measure on a \textbf{metric} space \((X,d)\), then the completion \(\bar{\mathscr{B}}(X)\) of the Borel \(\sigma\)-algebra
    with respect to \(\mu\) is called the \(\sigma\)-algebra of \(\mu\)-measurable sets. 
\end{definition}
\begin{remark}
    For \(\mu=\lambda_n\) on \(\mathbb{R}^n\) we talk about the \(\sigma\)-algebra of \textbf{Lebesgue measurable sets}. Instead of
    \(\bar{\lambda_n}\) we still write \(\lambda_n\) and call it the \textbf{Lebesgue measure}. A function \(f:\mathbb{R}^n\rightarrow \mathbb{C}\),
    measurable w.r.t. the \(\sigma\)-algebra of Lebesgue measurable sets is called the \textbf{Lebesgue measurable}.
\end{remark}
\emph{The following result shows that any Lebesgue measurable function coincides with a Borel function a.e.}
\begin{proposition}
    Assume \(\left( X,\mathscr{B},\mu \right)\) is a measure space and consider its completion \(\left( X,\bar{\mathscr{B}},\bar{\mu} \right)\).
    Assume \(f:X\rightarrow\mathbb{C}\) is \(\bar{\mathscr{B}}\)-measurable. Then there is a \(\mathscr{B}\)-measurable function 
    \(g:X\rightarrow\mathbb{C}\) s.t. \(f=g \ \bar{\mu}\)-a.e.
\end{proposition}

\section{Convergence Theorems and Their Applications (lecture 9, 8. Feb.)}
- To interchange limits and integrals in \textbf{Riemann integrals} one typically has to assume uniform convergence. <- The set of Riemann
integrable functions is somewhat limited, see theorem \ref{theorem:12_9}

\begin{theorem}[Generalization of Beppo Levi, monotone convergence]
    \quad
    
    \begin{enumerate}[label=(\roman*)]
        \item Let \((u_n)_{n\in\mathbb{N}}\subset \mathcal{L}^{1}(\mu)\) be s.t. \(u_1\leq u_2 \leq ...\) with limit 
        \(u\eqdef \sup_{n\in\mathbb{N}}u_n = \lim_{n\rightarrow\infty} u_n\). Then \(u\in\mathcal{L}^{1}(\mu)\) \textbf{iff} 
        \begin{eqnarray*}
            \sup\limits_{n\in\mathbb{N}}\int u_nd\mu <+\infty,
        \end{eqnarray*}
        in which case
        \begin{eqnarray*}
            \sup\limits_{n\in\mathbb{N}}\int u_n d\mu = \int\sup\limits_{n\in\mathbb{N}}u_n d\mu.
        \end{eqnarray*}
        \item Same thing only with a decreasing sequence ...\(>-\infty\) in which case
        \begin{eqnarray*}
            \inf\limits_{n\in\mathbb{N}}\int u_n d\mu = \int\inf\limits_{n\in\mathbb{N}}u_n d\mu.
        \end{eqnarray*}
    \end{enumerate}
\end{theorem}

\begin{theorem}[Lebesgue; dominated convergence]
    Let \((u_n)_{n\in\mathbb{N}}\subset\mathcal{L}^{1}(\mu)\) s.t.
    \begin{enumerate}[label=(\alph*)]
        \item \(|u_n|(x)\leq w(x)\), \(w\in\mathcal{L}^{1}(\mu)\),
        \item \(u(x) = \lim_{n\rightarrow\infty}u_n(x)\) exists in \(\bar{\mathbb{R}}\),
    \end{enumerate}
    then \(u\in\mathcal{L}^{1}(\mu)\) and we have
    \begin{enumerate}[label=(\roman*)]
        \item \(\lim\limits_{n\rightarrow\infty} \int \vert u_n - u\vert d\mu = 0\);
        \item \(\lim\limits_{n\rightarrow\infty} \int u_n d\mu = \int\lim\limits_{n\rightarrow\infty}u_n d\mu = \int ud\mu\);
    \end{enumerate}
\end{theorem}
\ifdetailed
\subsection*{Application 1: Parameter-Dependent Integrals}
- We are interested in questions of the sort, when is 
\begin{align*}
    U(t) \eqdef \int u(t,x)\mu(dx), \ t\in(a,b),
\end{align*}
again a smooth function of t? The answer involves interchange of limits and integration. Also, it turns out to better understand Riemann
integrability, we need the Lebesgue integral.
\begin{theorem}[continuity lemma]
    Let \(\emptyset \neq(a,b)\subset\mathbb{R}\) be a non-degenerate open interval and \(u:(a,b)\times X \rightarrow\mathbb{R}\) satisfy
    \begin{enumerate}[label=(\alph*)]
        \item \(x\mapsto u(t,x)\) is in \(\mathcal{L}^{1}(\mu)\) for every fixed \(t\in(a,b)\);
        \item \(t\mapsto u(t,x)\) is continuous for every fixed \(x\in X\);
        \item \(\vert u(t,x)\vert \leq w(x)\) for all \((t,x)\in (a,b)\times X\) and some \(w\in\mathcal{L}^{1}(\mu)\).
    \end{enumerate}
    Then the function \(U:(a,b)\rightarrow\mathbb{R}\) given by
    \begin{align} \label{eq:U(t)_tjohei}
        t\mapsto U(t)\eqdef \int u(t,x)\mu(dx)
    \end{align}
    is continuous.
\end{theorem}
\begin{theorem}[differentiability lemma]
    Let \(\emptyset\leq(a,b)\subset\mathbb{R}\) be a non-degenerate open interval and \(u:(a,b)\times X\rightarrow\mathbb{R}\) satisfy
    \begin{enumerate}[label=(\alph*)]
        \item Same
        \item Same 
        \item \(\vert\partial_t u(t,x)\vert \leq w(x)\) for all \((t,x)\in(a,b)\times X\) and some \(w\in\mathcal{L}^{1}(\mu)\).
    \end{enumerate}
    Then the function in \ref{eq:U(t)_tjohei} is differentiable and its derivative is
    \begin{align}
        \frac{d}{dt}U(t) = \frac{d}{dt}\int u(t,x)\mu(dx) = \int \frac{\partial}{\partial t}u(t,x)\mu(dx).
    \end{align}
\end{theorem}
\fi

\subsection*{Application 2: Riemann vs Lebesgue Integration}
Consider only \(\left(X,\mathscr{A},\mu\right) = \left(\mathbb{R},\mathscr{B}(\mathbb{R}),\lambda\right)\).
\ifdetailed
\begin{definition}[The Riemann Inegral]
    Consider on the finite interval \([a,b]\subset\mathbb{R}\) the partition
    \begin{align}
        \Pi \eqdef \left\{ a=t_0<t_1<...<t_k<b \right\}, k=k(\Pi),
    \end{align}
    and introduce
    \begin{eqnarray}
        S_{\Pi}[u] \eqdef \sum\limits_{i=1}^{k(\Pi)}m_i(t_i-t_{i-1}), &\quad& m_i \eqdef\inf\limits_{x\in[t_{i-1}, t_i]}u(x), \\
        S^{\Pi}[u] \eqdef \sum\limits_{i=1}^{k(\Pi)}M_i(t_i-t_{i-1}), &\quad& M_i \eqdef\sup\limits_{x\in[t_{i-1}, t_i]}u(x). \\
    \end{eqnarray}
    A bounded function \(u:[a,b]\rightarrow\mathbb{R}\) is said to be \textbf{Riemann integrable} if the values
    \begin{align}
        \int\limits_{\_} u \eqdef \sup\limits_{\Pi}S_{\Pi}[u] = \inf\limits_{\Pi}S^{\Pi}[u] =\mathrel{\mathop:} \int\limits^{\_} u
    \end{align}
    coincide and are finite. Their common value is called the \textbf{Riemann integral} of \(u\) and denoted by 
    \((R)\int\limits_{a}^{b}u(x)dx\) or \(\int\limits_{a}^{b}u(x)dx\).
\end{definition}
\fi 
\begin{theorem}
    Let \(u:[a,b]\rightarrow\mathbb{R}\) be a \textbf{measurable} and \textbf{Riemann integrable} function. Then
    \begin{align}
        u\in\mathcal{L}^{1}(\lambda) \text{ and } \int\limits_{[a,b]}ud\lambda = \int\limits_{a}^{b}u(x)dx.
    \end{align}
\end{theorem}
\begin{theorem} \label{theorem:12_9}
    Let \(u:[a,b] \rightarrow \mathbb{R}\) be a bounded function, it is Riemann integrable \textbf{iff} the points in \(\left(a,b\right)\) where
    \(u\) is discontinuous are a (subset of) Borel measurable null set.
\end{theorem}

\subsection*{Improper Riemann Integrals}
- The Lebesgue integral extends the (\emph{proper}) Riemann integral. However, there is a further extension of the Riemann integral which
cannot be captured by Lebesgue's theory. \(u\) is Lebesgue integrable \emph{iff} \(\vert u\vert\) ha finite Lebesgue integral. <- The Lebesgue
integral does not respect sign-changes and cancellations. However, the following \emph{improper Riemann integral} does:
\begin{eqnarray}
    (R)\int\limits_{0}^{\infty}u(x)dx\eqdef \lim\limits_{n\rightarrow\infty}(R)\int\limits_{0}^{a}u(x)dx.
\end{eqnarray}
\begin{corollary}
    Let \(u:[0,\infty) \rightarrow\mathbb{R}\) be a measurable, Riemann integrable function for every interval \([0,N], \ N\in\mathbb{N}\). 
    Then \(u\in\mathcal{L}^{1}[0,\infty)\)
    \textbf{iff}
    \begin{align}
        \lim\limits_{N\rightarrow\infty}(R)\int\limits_{0}^{N}\vert u(x)\vert dx < \infty.
    \end{align}
    In this case, \((R)\int_{0}^{\infty}u(x)dx = \int_{[0,\infty)}ud\lambda\)
\end{corollary}
\ifdetailed
\textbf{Example} of a function which is \emph{improperly Riemann integrable} but \textbf{not} \emph{Lebesgue integrable}:
\begin{align}
    f(x) = \frac{\sin(x)}{x}.
\end{align}
\fi 
\begin{proposition}[appearing as example 12.13 in Schilling]
    Let \(f_{\alpha}(x)\eqdef x^{\alpha}, x>0\) and \(\alpha\in\mathbb{R}\). Then 
    \begin{enumerate}[label=(\roman*)]
        \item \(f_(\alpha)\in\mathcal{L}^{1}(0,1)\Leftrightarrow \alpha > -1\).
        \item \(f_(\alpha)\in\mathcal{L}^{1}[1,\infty)\Leftrightarrow \alpha < -1\).
    \end{enumerate}
\end{proposition}