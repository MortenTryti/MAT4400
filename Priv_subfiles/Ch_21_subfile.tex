

\section{Duals of $\mathbf{L^p}$-spaces}
Assume $(X,\B,\mu)$ is a measure space, $1\leq p<\infty$. What is the dual of 
$L^p(X,d\mu)$? When does a measurable function $g:X\rightarrow \C$ define a bounded linear functional on $L^p(X,d\mu)$ by 
$$\phi(f)=\int_X fgd\mu?$$

\begin{theorem}(\underline{\textbf{Young's inequality}})
    Assume $f:[0,a]\rightarrow [0,b]$ is a strictly increasing continuous functions, $f(0)=0$, $f(a)=b$. Then for all $s\in[0,a]$ and $t\in[0,b]$
    we have $$st\leq \int_0^s f(x)dx + \int_0^t f^{-1}dy dy$$
    and the equality holds if and only if $t=f(s)$.
    
\end{theorem}
If we apply this to $f(s)=s^{p-1}$. Then $f^{-1}(t)=t^{q-1}$, where $q$ is the \underline{Hölder conjugate} of $p$. $(p-1)(q-1)=1$, that is $$\frac{1}{p}+\frac{1}{q}=1.$$
We get $$st \leq \int_0^s x^{p-1}dx + \int_0^t y^{q-1}dy = \frac{s^p}{p}+\frac{t^q}{q}.$$
\begin{theorem}(\underline{\textbf{Hölder's inequality}})
    If $f\in L^p(X,d\mu)$, $g\in L^q(X,d\mu)$, $1<p<\infty$ and $1/p + 1/q = 1$. Then 
    $$fg \in L^1(X,d\mu)\hspace{4mm}\text{and}\hspace{4mm} ||fg||_1 \leq ||f||_p||g||_q.$$
\end{theorem}


It follows that every $g\in L^q(X,d\mu)$ defines a bounded linear functional 
$$l_g:L^p(X,d\mu)\rightarrow \C,\hspace{4mm} l_g(f) = \int_X fg d\mu,\hspace*{3mm}\text{and}\hspace{3mm} ||l_g||\leq ||g||_q.$$
The same makes sense for $p=1$, $q=\infty$ and $p=\infty$, $q=1$, when $\mu$ is $\sigma$-finite as 
$$\int_X|fg|d\mu\leq \int_X|f|d\mu ||g||_\infty = ||f||_1||g||_\infty.$$

\begin{lemma}
    Assume $1\leq p\leq\infty$, $1/p + 1/q = 1$ and $\mu$ is $\sigma$-finite if $p=1$ or $p=\infty$. For $g\in L^q(X,d\mu)$ consider $l_g\in L^p(X,d\mu)^*$. Then 
    $$||l_g|| = ||g||_q.$$
\end{lemma}


Therefore we can view $L^q(X,d\mu)$ as a subspace of $L^p(X,d\mu)^*$ using the isometric embedding
$$L^q(X,d\mu)\hookrightarrow L^p(X,d\mu),\hspace{2mm}g\mapsto l_g.$$ 
\begin{theorem}
    Assume $(X,\B,\mu)$ is a $\sigma$-finite measure space, $1\leq p<\infty$, $1/p+1/q=1$. Then 
    $$L^p(X,d\mu)^* = L^q(X,d\mu).$$
\end{theorem}

\begin{remark}
    This is usually not true for $p=\infty$.
\end{remark}

\section*{Dual of $C(X)$ spaces (lecture 4.april)}
When $\mu$ is sigma finite we know that \[L^p(X,d\mu)^* = L^q(X,d\mu)\]
when $1\leq p<\infty$ and $p^{-1}+q^{-1}=1$. However, what is the dual space of $C(X)$?
Assume $(X,d)$ is a locally compact metric space. Consider the space $C_c(X)$ of continuous functions with compact support on $X$.
A linear functional 
\[\phi:C_c(X)\rightarrow \C\] is called \underline{positive} if $\phi(f)\geq 0$ for $f\geq 0$.
\begin{example}
    Assume $\mu$ is a Borel measure on $X$ s.t. $\mu(K)<\infty$ for every compact $K\subset X$. Then \[\phi(f)=\int_X fd\mu\] defines a positive linear functional on $C_c(X)$.
\end{example}

\begin{theorem}(\underline{\textbf{Riesz-Markov}})
    Assume $(X,d)$ is a locally compact metric space and $\phi:C_c(X)\rightarrow \C$ is a positive linear functional. Then there is a Borel measure $\mu$ such that $\mu(K)<\infty$ for every compact $K\subset X$ and 
    \[\phi(f)=\int_Xfd\mu,\hspace{4mm}\forall f\in C_c(X).\]
    Such a measure is unique if $X$ is separable.
\end{theorem}
To prove this we need some other results
\begin{lemma}(\underline{\textbf{Urysohn's lemma}})
    Assume $(X,d)$ is a metric space, $A,B\subset X$ are disjoint closed subsets. Then there is a continuous function 
    $f:X\rightarrow [0,1]$ s.t. $\left.f\right|_A=1$ , $\left.f\right|_B=0$. 
\end{lemma}

\begin{lemma}
    Assume $(X,d)$ is a compact metric space, assume we have a finite open cover $U=(U_i)_{i=1}^n$ of $X$. Then there exists
    continuous functions $\rho_1,\ldots,\rho_n$ such that \[\text{supp}(\rho_i)\subset U_i,\hspace{4mm} \rho_i\geq 0,\hspace{4mm} \sum_{i=1}^n \rho_i = 1.\]
    Every such function is called a partition of unity \underline{subordinate} to $U$.

\end{lemma}

\begin{remark}
    Without separability of $X$ we can assume that the uniqueness holds within the class of Borel measures $\mu$ s.t.
    \begin{enumerate}
        \item $\mu(K)<\infty$ for every compact $K$
        \item $\mu$ is outer regular $\mu(A) = \inf_{A\subset U,\hspace{1mm}U\text{ open}}\mu(U)$
        \item $\mu$ is inner regular on open sets, $\mu(U) = \sup_{K\subset U,\hspace{1mm}K\text{ compact}}\mu(K)$
    \end{enumerate} 
    Such measures are called \underline{Radon measures}.
\end{remark}

As an application we will describe $C(X)^*$ for compact metric spaces $X$ in terms of measures. Denote by $M(X)$ the space of complex Borel measures on X.
For every $\nu\in M(X)$ we want to make sense of

\[\int_X fd\nu \tmm\text{for}\tmm f\in C(X).\]
It is enough to consider finite signed measures, as we can then define 
\[ \int_X f d\nu = \int_X fd(Re\nu) +i\int_X fd(Im\nu).\]
So assume that $\nu$ is a finite signed measure. Then $\nu = \mu_1-\mu_2$ for positive measures and we can define 
\[\int_X fd\nu = \int_X fd\mu_1 - \int_X fd\mu_2.\]
This is well defined, since if \[\nu = \mu_1-\mu_2 = \omega_1-\omega_2,\]
then $\mu_1+\omega_2 = \mu_2+\omega_1$ and 
\[\int_X fd\mu_1 + \int_X fd\omega_2 = \int_X fd\mu_2 + \int_Xfd\omega_1.\]
Thus, every $\nu\int M(X)$ defines a linear functional $\phi_\nu:C(X)\rightarrow \C$ by \[ \phi_\nu(f) = \int_X fd\nu\] and the map 
$\nu\mapsto \phi_\nu$ is linear. 

\begin{lemma}
    If $\nu\in M(X)$ and $d\nu = gd|\nu|$ is its polar decomposition, then 
    \[\int_X fd\nu = \int_X fgd|\nu|,\tmm \forall f\in C(X).\]
\end{lemma}


\begin{lemma}
    For every $\nu\in M(X)$, the linear functional $\phi_\nu$ is bounded and \[||\phi_\nu|| = |\nu|(X).\]
    Recall that the norm on $C(X)$ is \[||f||=\sup_{x\in X}|f(x)|.\]
\end{lemma}


Now consider $(X,d)$ to be a compact metric space. Denote $M(X)$ to be the set of complex Borel measures on $X$.
\[M(X)\rightarrow C(X)^*,\tmm \nu\mapsto \phi_\nu,\tmm \phi_\nu(f)=\int_X fd\nu,\tmm ||\phi_\nu|| = |\nu|(X).\]


\begin{lemma}
    $\quad$
    \begin{enumerate}
        \item If $\phi:C(X)\rightarrow \C$ is positive, then it is bounded and $||\phi|| = \phi(1)$.
        \item The positive linear functional on $C(X)$ span $C(X)^*$.
    \end{enumerate}
\end{lemma}

\begin{theorem}
    Assume $(X,d)$ is a compact metric space. Then the map 
    \[M(X)\rightarrow C(X)^*,\tmm \nu\mapsto \phi_\nu,\tmm \phi_\nu(f) = \int_X fd\mu,\]
    is a linear isomorphism.
\end{theorem}

\begin{remark}
    $\quad$
    \begin{enumerate}
        \item As a byproduct we see that $M(X)$ is a Banach space with norm $||\nu|| = |\nu|(X)$.
        \item Assume $(X,d)$ is a locally compact metric space 
        \[C_0(X) \equiv \text{the space of all continuous functions vanishing at infinity}\]
        \[ = \left\{f:X\rightarrow \C \omm:\omm f \text{ is continuous},\tmm f(x)\rightarrow 0,\text{ as }x\rightarrow 0\right\}.\]
        Where $"f(x)\rightarrow 0,\text{ as }x\rightarrow 0"$ means that $\forall \epsilon>0\exists$ compact $K\subset X$ s.t. $|f(K)|<\epsilon$ $\forall x\in K^c$.
        This is a Banach space with norm $||f|| = \sup_{x\in X}|f(x)|$ and $C_c(X)$ is dense in $C_0(X)$.
        One can prove that if $X$ is separable, then we again have a linear isomorphism $M(X)\xrightarrow{\sim} C_0(X)^*$, $\nu\mapsto \phi_\nu,\tmm \phi_\nu(f)=\int_X fd\nu$.
        For general $X$ we simply have $M_{reg}(X)\xrightarrow{\sim}C_0(X)^*$, where $M_{reg}(X)$ is the space of regular complex Borel measures on $X$. Here regularity of $\nu$ means that $|\nu|$ is regular.
    \end{enumerate}
\end{remark}