\documentclass{article}


% allows special characters (including æøå)
\usepackage[utf8]{inputenc}
%\usepackage[english]{babel}
\usepackage{dsfont}
\usepackage{subfiles}
\usepackage{physics,amssymb}  % mathematical symbols (physics imports amsmath)
\include{amsmath}
\usepackage{graphicx}         % include graphics such as plots
\usepackage{xcolor}           % set colors
\usepackage{hyperref}         % automagic cross-referencing (this is GODLIKE)
\usepackage{listings}         % display code
\usepackage{subfigure}        % imports a lot of cool and useful figure commands
\usepackage{float}
%\usepackage[section]{placeins}
\usepackage{algorithm}
\usepackage[noend]{algpseudocode}
\usepackage{subfigure}
\usepackage{tikz}
\usepackage{cleveref} % for \cref
\usepackage{enumitem} % to enumerate with a), b), ... : [label=(\alph*)] 
\usepackage{cancel}
\usepackage{slashed}
\usepackage{amsthm}
\usepackage{mathrsfs}
\usepackage{comment}
\usepackage{bbm} % Allows for the characteristic function of a set

\newtheorem{theorem}{Theorem}[section]
\newtheorem{lemma}[theorem]{Lemma}
\newtheorem{properties}[theorem]{Properties}
\newtheorem{corollary}[theorem]{Corollary}
\newtheorem{proposition}[theorem]{Proposition}
\newtheorem*{remark}{Remark}
\newcommand{\eqdef}{\mathrel{\mathop:}=}
\theoremstyle{definition}
\newtheorem{definition}[theorem]{Definition}
\usepackage{stmaryrd}



\usepackage{geometry}
 \geometry{
 a4paper,
 total={170mm,257mm},
 left=20mm,
 top=20mm,
 }

\usetikzlibrary{quantikz}

% defines the color of hyperref objects
% Blending two colors:  blue!80!black  =  80% blue and 20% black


\newcommand{\A}{\mathscr{A}}
\newcommand{\R}{\mathbb{R}}
\newcommand{\C}{\mathbb{C}}
\newcommand{\omm}{\hspace{1mm}}
\newcommand{\tmm}{\hspace{2mm}}
\newcommand{\one}{\mathds{1}}
\newcommand{\E}{\mathcal{E}}
\newcommand{\N}{\mathbb{N}}
\newcommand{\M}{\mathcal{M}}
\newcommand{\B}{\mathscr{B}}
\newcommand{\Borel}[1]{\mathscr B(\mathbb R^{#1})}
\newcommand{\map}[3]{#1:#2\rightarrow #3}
\newcommand{\highlight}[2][yellow]{\mathchoice%
  {\colorbox{#1}{$\displaystyle#2$}}%
  {\colorbox{#1}{$\textstyle#2$}}%
  {\colorbox{#1}{$\scriptstyle#2$}}%
  {\colorbox{#1}{$\scriptscriptstyle#2$}}}
\newcommand*{\itemtag}[1]{\hfill (\emph{#1})}

\lstset{frame=tb,
  language=c++,
  aboveskip=3mm,
  belowskip=3mm,
  showstringspaces=false,
  columns=flexible,
  basicstyle={\small\ttfamily},
  numbers=none,
  numberstyle=\tiny\color{gray},
  keywordstyle=\color{blue},
  commentstyle=\color{dkgreen},
  stringstyle=\color{orange},
  breaklines=true,
  breakatwhitespace=true,
  tabsize=4
}
\hypersetup{
    colorlinks,
    linkcolor={red!50!black},
    citecolor={blue!50!black},
    urlcolor={blue!80!black}}
%% USEFUL LINKS:
%%
%%   UiO LaTeX guides:        https://www.mn.uio.no/ifi/tjenester/it/hjelp/latex/
%%   mathematics:             https://en.wikibooks.org/wiki/LaTeX/Mathematics

%%   PHYSICS !                https://mirror.hmc.edu/ctan/macros/latex/contrib/physics/physics.pdf

%%   the basics of Tikz:       https://en.wikibooks.org/wiki/LaTeX/PGF/Tikz
%%   all the colors!:          https://en.wikibooks.org/wiki/LaTeX/Colors
%%   how to draw tables:       https://en.wikibooks.org/wiki/LaTeX/Tables
%%   code listing styles:      https://en.wikibooks.org/wiki/LaTeX/Source_Code_Listings
%%   \includegraphics          https://en.wikibooks.org/wiki/LaTeX/Importing_Graphics
%%   learn more about figures  https://en.wikibooks.org/wiki/LaTeX/Floats,_Figures_and_Captions
%%   automagic bibliography:   https://en.wikibooks.org/wiki/LaTeX/Bibliography_Management  (this one is kinda difficult the first time)
%%   REVTeX Guide:             http://www.physics.csbsju.edu/370/papers/Journal_Style_Manuals/auguide4-1.pdf
%%
%%   (this document is of class "revtex4-1", the REVTeX Guide explains how the class works)


%%%%%%%%%%%%% CREATING THE pdf FILE USING LINUX IN THE TERMINAL %%%%%%%%%%%%%
%% pdflatex filename.tex && filename.tex && open filename.pdf

%%%%%%%%%%%%%%%%%%%%%%%%%% USING FOOTNOTE COMMAND: %%%%%%%%%%%%%%%%%%%%%%%%%%
    %% pdflatex filename.tex && bibtex filename.tex && pdflatex filename.tex && pdflatex filename.tex && open filename.pdf


\newif\ifdetailed
\detailedtrue % Detailed version
% \detailedfalse % Only important stuff version
%%%% Usage %%%%:
% \ifdetailed
% Detailed thing, for example:
% \begin{align}
%   U(t) = \int u(x,t)d\mu
% \end{align}
% \fi 


\setcounter{section}{17}
\setcounter{theorem}{17}

\begin{document}
\section{Dense subspaces of $L^p$}
\begin{theorem}
    Assume $(X,d)$ is a metric space, $\mu$ is a Borel measure on $X$ s.t. $\mu(B_R(x))<\infty,\tmm \forall x\in X$ and 
    $\forall R>0$, $1\leq p\leq \infty$. Then the bounded continuous functions on $X$ with bounded support form a dense subspace of $L^p(X,d\mu)$.
    (Where by bounded support we mean that $f$ is zero outside $B_R(x)$ for some $x$ and $R>0$.)
\end{theorem}

If $X$ is locally compact, then by $C_c(X)$ we denote the space of continuous functions on $X$ with compact support.

\begin{theorem}
    Assume $(X,d)$ is a separable, so it has a dense subset, locally compact metric space, $\mu$ is a Borel measure on $X$ s.t. $\mu(K)<\infty$ for every compact $K\subset X$, $|\leq P<\infty$. Then $C_c(X)$ is dense in $L^p(X,d\mu)$.
\end{theorem}

\begin{remark}
    Theorem 17.8 in the book is wrong.
\end{remark}

\begin{remark}
    These results do not extent to $p=\infty$.
\end{remark}
For $X=\R^n$, either theorem implies that if $\mu$ is a Borel measure on $\R^n$, s.t. $\mu(B_R(x))<\infty$, $\forall x$, $\forall R>0$, then
$C_c(\R^n)$ is dense in $L^p(\R^n,d\mu)$, $1\leq p<\infty$. Later we will prove that $C_c^\infty (\R^n)$ is still dense in $L^p(\R^n,d\mu)$.
 For $\mu=\lambda_n$ we write $L^p(\R^n)$ instead of $L^p(\R^n,d\lambda_n)$.



\section{Modes of convergence}
\begin{theorem}(\underline{\textbf{Egorov}})

    Assume $(X,\B,\mu)$ is a measure space, $\mu(X)<\infty$. Assume $f_n,f:X\rightarrow\C$ are measurable functions and 
    $f_n\rightarrow f$ a.e. Then $\forall \epsilon>0$ there is $X_\epsilon \in\B$ s.t. $\mu(X_\epsilon)<\epsilon$ and $f_n\rightarrow f$ uniformly on $X\backslash X_\epsilon$. 
\end{theorem}

\begin{enumerate}
    \item For measurable functions $f_n,f$, we write $f_n\rightarrow f$ in the $p-th$ mean, excluding $p=\infty$, if $\lim_{n\rightarrow\infty}||f_n-f||_p=0$.
    \item For $p=1$ we sat that $f_n\rightarrow f$ in \underline{mean} and for $p=2$ we say that $f_n\rightarrow f$ in \underline{quadratic mean}.
    \item We say that $f_n\rightarrow f$ in \underline{measure} if $\lim_{n\rightarrow\infty}\mu(\left\{x:|f_n(x)-f(x)|\geq \epsilon\right\})=0$, $\forall \epsilon>0$.
\end{enumerate}

\begin{theorem}
    Assume $(X,\B,\mu)$ is a measure space, $1\leq p<\infty$, $f_n,f\rightarrow \C$ are measurable functions. Then
    \begin{enumerate}
        \item if $f_n\rightarrow f$ in the $p$-th mean, then $f_n\rightarrow f$ in measure 
        \item if $f_n\rightarrow f$ in measure, then there is a subsequence $(f_n)_{n=1}^\infty$ s.t. $f_{n_k}\rightarrow f$ a.e.
        \item if $f_n\rightarrow f$ a.e. and $\mu(X)<\infty$, then $f_n\rightarrow f$ in measure.
    \end{enumerate}
    In particular, if $f_n\rightarrow f$ in the $p$-th mean, then $f_{n_k}\rightarrow f$ a.e. for a subsequence $(f_{n_k})_k$.
\end{theorem}

\end{document}