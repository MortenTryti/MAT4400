\documentclass{article}


% allows special characters (including æøå)
\usepackage[utf8]{inputenc}
%\usepackage[english]{babel}
\usepackage{dsfont}
\usepackage{subfiles}
\usepackage{physics,amssymb}  % mathematical symbols (physics imports amsmath)
\include{amsmath}
\usepackage{graphicx}         % include graphics such as plots
\usepackage{xcolor}           % set colors
\usepackage{hyperref}         % automagic cross-referencing (this is GODLIKE)
\usepackage{listings}         % display code
\usepackage{subfigure}        % imports a lot of cool and useful figure commands
\usepackage{float}
%\usepackage[section]{placeins}
\usepackage{algorithm}
\usepackage[noend]{algpseudocode}
\usepackage{subfigure}
\usepackage{tikz}
\usepackage{cleveref} % for \cref
\usepackage{enumitem} % to enumerate with a), b), ... : [label=(\alph*)] 
\usepackage{cancel}
\usepackage{slashed}
\usepackage{amsthm}
\usepackage{mathrsfs}
\usepackage{comment}
\usepackage{bbm} % Allows for the characteristic function of a set

\newtheorem{theorem}{Theorem}[section]
\newtheorem{lemma}[theorem]{Lemma}
\newtheorem{properties}[theorem]{Properties}
\newtheorem{corollary}[theorem]{Corollary}
\newtheorem{proposition}[theorem]{Proposition}
\newtheorem*{remark}{Remark}
\newcommand{\eqdef}{\mathrel{\mathop:}=}
\theoremstyle{definition}
\newtheorem{definition}[theorem]{Definition}
\newtheorem{example}[theorem]{Example}
\usepackage{stmaryrd}



\usepackage{geometry}
 \geometry{
 a4paper,
 total={170mm,257mm},
 left=20mm,
 top=20mm,
 }

\usetikzlibrary{quantikz}

% defines the color of hyperref objects
% Blending two colors:  blue!80!black  =  80% blue and 20% black


\newcommand{\A}{\mathscr{A}}
\newcommand{\R}{\mathbb{R}}
\newcommand{\C}{\mathbb{C}}
\newcommand{\omm}{\hspace{1mm}}
\newcommand{\tmm}{\hspace{2mm}}
\newcommand{\one}{\mathds{1}}
\newcommand{\E}{\mathcal{E}}
\newcommand{\N}{\mathbb{N}}
\newcommand{\M}{\mathcal{M}}
\newcommand{\B}{\mathscr{B}}
\newcommand{\Borel}[1]{\mathscr B(\mathbb R^{#1})}
\newcommand{\map}[3]{#1:#2\rightarrow #3}
\newcommand{\highlight}[2][yellow]{\mathchoice%
  {\colorbox{#1}{$\displaystyle#2$}}%
  {\colorbox{#1}{$\textstyle#2$}}%
  {\colorbox{#1}{$\scriptstyle#2$}}%
  {\colorbox{#1}{$\scriptscriptstyle#2$}}}
\newcommand*{\itemtag}[1]{\hfill (\emph{#1})}

\lstset{frame=tb,
  language=c++,
  aboveskip=3mm,
  belowskip=3mm,
  showstringspaces=false,
  columns=flexible,
  basicstyle={\small\ttfamily},
  numbers=none,
  numberstyle=\tiny\color{gray},
  keywordstyle=\color{blue},
  commentstyle=\color{dkgreen},
  stringstyle=\color{orange},
  breaklines=true,
  breakatwhitespace=true,
  tabsize=4
}
\hypersetup{
    colorlinks,
    linkcolor={red!50!black},
    citecolor={blue!50!black},
    urlcolor={blue!80!black}}
%% USEFUL LINKS:
%%
%%   UiO LaTeX guides:        https://www.mn.uio.no/ifi/tjenester/it/hjelp/latex/
%%   mathematics:             https://en.wikibooks.org/wiki/LaTeX/Mathematics

%%   PHYSICS !                https://mirror.hmc.edu/ctan/macros/latex/contrib/physics/physics.pdf

%%   the basics of Tikz:       https://en.wikibooks.org/wiki/LaTeX/PGF/Tikz
%%   all the colors!:          https://en.wikibooks.org/wiki/LaTeX/Colors
%%   how to draw tables:       https://en.wikibooks.org/wiki/LaTeX/Tables
%%   code listing styles:      https://en.wikibooks.org/wiki/LaTeX/Source_Code_Listings
%%   \includegraphics          https://en.wikibooks.org/wiki/LaTeX/Importing_Graphics
%%   learn more about figures  https://en.wikibooks.org/wiki/LaTeX/Floats,_Figures_and_Captions
%%   automagic bibliography:   https://en.wikibooks.org/wiki/LaTeX/Bibliography_Management  (this one is kinda difficult the first time)
%%   REVTeX Guide:             http://www.physics.csbsju.edu/370/papers/Journal_Style_Manuals/auguide4-1.pdf
%%
%%   (this document is of class "revtex4-1", the REVTeX Guide explains how the class works)


%%%%%%%%%%%%% CREATING THE pdf FILE USING LINUX IN THE TERMINAL %%%%%%%%%%%%%
%% pdflatex filename.tex && filename.tex && open filename.pdf

%%%%%%%%%%%%%%%%%%%%%%%%%% USING FOOTNOTE COMMAND: %%%%%%%%%%%%%%%%%%%%%%%%%%
    %% pdflatex filename.tex && bibtex filename.tex && pdflatex filename.tex && pdflatex filename.tex && open filename.pdf


\newif\ifdetailed
\detailedtrue % Detailed version
% \detailedfalse % Only important stuff version
%%%% Usage %%%%:
% \ifdetailed
% Detailed thing, for example:
% \begin{align}
%   U(t) = \int u(x,t)d\mu
% \end{align}
% \fi 


\setcounter{section}{4}
\setcounter{theorem}{4}

\begin{document}
\section{Hahn-Banach theorem}

\begin{theorem}(\underline{\textbf{Hahn-Banach theorem}})
    Assume $V$ is a real vector space, $V_0\subset V$ a subspace, $\phi:V\rightarrow \R$ a convex function and $f:V_0\rightarrow \R$ a linear functional
     s.t. $f\leq \phi$ on $V_0$. Then $f$ can be extended to a linear functional $F$ on $V$ s.t. $F\leq\phi$.
    
\end{theorem}

\begin{theorem}(\underline{\textbf{Hahn-Banach theorem 2}})
    Assume $V$ is a real or complex vector space, $p$ is a seminorm on $V$, $V_0\subset V$ a subspace and $f$ is a linear
    functional on $V_0$ s.t. $|f(x)|\leq p(x)$ $\forall x\in V_0$. Then $f$ can be extended to a linear functional $F$ on $V$ s.t.
    $|F(x)|\leq p(x)$ $\forall x\in V$.
\end{theorem}

\begin{corollary}
    Assume $V$ is a normed vector space (real or complex), $V_0\subset V$ a subspace and $F\in V_0^*$. Then there is $F\in V^*$ s.t.
    $\left.F\right|_{V_0}=f$ and $||F||=||f||$.
\end{corollary}

\begin{corollary}
    Assume $V$ is a normed space and $x\in V, x\neq 0$. Then there is $F\in V^*$ s.t. $||F||=1$ and $F(x) = ||x||$.
\end{corollary}
Such an $F$ is called a supporting functional of $x$.

\begin{definition}
    A normed space $V$ is called \underline{reflexive} if $V^{**}=V$.
\end{definition}
\begin{remark}
    This is stronger than requiring $V\cong V^{**}$.
\end{remark}
\begin{example}
    \begin{enumerate}
        \item Every f.d. normed vector space $V$ is reflexive for dimensional reasons $\dim V^{**} = \dim V^* = \dim V$.
        \item Every Hilbert space $H$ is reflexive. By Riesz' theorem every bounded linear functional $f$ on $\overline H$ has the form
        $$f(\bar x) = (\bar x,\bar y)=(x,y)\tmm y\in H,$$
        which means that $f=y$ in $H^{**}$. As we will later see, the spaces $L^p(X,d\mu)$, with $\mu$ $\sigma$-finite and $1<p<\infty$, are reflexive. The
        spaces $L^1(X,d\mu)$ and $L^\infty(X,d\mu)$ are usually not reflexive.
    \end{enumerate}
\end{example}
\end{document}